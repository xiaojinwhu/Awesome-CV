\documentclass[11pt,a4paper]{moderncv}  
 
 % possible options include font size ('10pt', '11pt' and '12pt'), paper size ('a4paper', 'letterpaper', 'a5paper', 'legalpaper', 'executivepaper' and 'landscape') and font family ('sans' and 'roman')
 
 \moderncvstyle[left,shortrules]{banking}                             % style options are 'casual' (default), 'classic', 'banking', 'oldstyle' and 'fancy'
 \moderncvcolor{blue}                               % color options 'black', 'blue' (default), 'burgundy', 'green', 'grey', 'orange', 'purple' and 'red'
 \usepackage[details,left]{moderncvheadi}
 \usepackage[top=1.5cm,bottom=1.2cm,left=2cm,right=2cm]{geometry}
 %\setlength{\hintscolumnwidth}{3cm}                % if you want to change the width of the column with the dates
 %\setlength{\makecvtitlenamewidth}{10cm}           % for the 'classic' style, if you want to force the width allocated to your name and avoid line breaks. be careful though, the length is normally calculated to avoid any overlap with your personal info; use this at your own typographical risks...

 \usepackage{xeCJK}
 \usepackage{fontspec}
 \usepackage{fontawesome}
 
 
  \usepackage{color}
 \definecolor{darkgray}{HTML}{333333}


 
% \setCJKmainfont{Hiragino Sans GB}
% \setmainfont{Hiragino Sans GB}
\setCJKmainfont{SourceHanSansCN}[Path=fonts/, UprightFont=*-Regular.otf, BoldFont=*-Bold.otf]
\setmainfont{Roboto}[Path=fonts/, UprightFont=*-Regular.ttf, BoldFont=*-Bold.ttf]
 
 \name{肖劲}{}
 \title{ }
 \photo[64pt][0.5pt]{picture} 
 \phone[mobile]{170-9183-3201}
 \email{xiaojinwhu0@gmail.com}
 %\extrainfo{\faicon{bookmark-o} 湖南省张家界市|中共党员}
 %\homepage{xiaojin.online}
 \address{\faicon{map-marker} 上海市徐汇区}{\faicon{graduation-cap} 复旦大学硕士}
 \quote{求职意向:数据/深度学习相关职位}
 
 \begin{document}
 
 \makecvtitle
 
  \section{个人简介}
 知名高校医学和生物学背景,以第一作者发表过SCI论文,一年产品经验,熟悉Python,爬虫,熟悉数据ETL,了解机器学习及自然语言处理基本方法。热爱学习并能快速学习,涉猎广泛,搭建过个人博客。在Udacity上完成深度学习基石的学习。
  
 \section{教育背景}

  \cventry{2013-2016}{理学硕士}{复旦大学}{生物化学与分子生物学专业}{}{
  	\begin{itemize}
  		\item 良好的科研背景和英文论文阅读能力
  	\end{itemize}
  	}
  \cventry{2008--2013}{医学学士}{武汉大学}{医学检验专业}{}{
  	\begin{itemize}
  		\item 高等数学,医学统计学
  	\end{itemize}
  }

\section{项目经验}
 \cventry{2016--至今}{产品经理\& Python工程师}{思路迪医疗科技有限公司}{}{}{
 	 \begin{itemize}
 	 	\item \textbf{文献自动查询系统Python开发及文本分析}
 	 	\begin{itemize}
 	 		\item 使用Python基于Requests,BeautifulSoup,Selenium 等爬取Google Scholar, NCBI网站。使用异步IO(asyncio)实现批量PDF文件下载,大幅提高下载效率,下载文献2千余篇。
 	 		\item 从头搭建生物医学文本库,并进行文本预处理形成领域内的语料库。 
 	 		编写词向量(Word2vec)模块,训练模型,生成词向量。 
 	 		\item 基于tf-idf算法(sklearn)提取文献单词权重。根据余弦相似性构建文献评分推荐系统。
 	 	\end{itemize}
  	    \item \textbf{生物样本库、临床肿瘤数据采集产品设计及数据分析}
  	\begin{itemize}
  		\item 主导设计基于云的生物样本库管理软件,帮助公司管理样本3万多个,将样本移交借出等过程转移到线上。将旧系统(Excel)数据进行清洗产品数据库。
  		\item 完成超过700个肿瘤数据元的定义,整理许多个临床领域字典。独立制作肿瘤数据3大产品宣传册(Indesign),初次打印超200 份,并在公司展会上派发。 
  		\item 基于医院随访数据使用R进行数据分析及数据可视化,并做了简单的数据挖掘(Apriori关联分析,生存曲线)。
  	\end{itemize}
  	    \item \textbf{生产系统数据仓库ETL工程}
  	\begin{itemize}
  		\item 对公司旧系统数据进行清洗,整理。使用开源工具Kettle搭建公司内部生产平台数据库数据ETL工程,部署在服务器上并实现数据定时ETL。 
  		\item 响应上级,多次完成数据整理等其他需求,提供tableau可视化意见。  	
 	 \end{itemize}
\end{itemize}
  }
   \cventry{2017--2017}{在线学习}{Udacity深度学习基石}{}{}{
   	\begin{itemize}
   		\item 基于tensorflow对CIFAR-10数据集进行分类,熟悉卷积神经网络的构建,优化(均一化,池化,Dropout)等方法。
   		\item 熟悉词嵌入,word2vec。基于LSTM构建生成电视剧剧本的RNN模型。基于seq2seq方法完成语言翻译项目。
   	\end{itemize}
 	}
   
% \section{实习实践}
% \cventry{2015--2016}{主要参与人}{肿瘤糖生物学}{}{}{
% 	 \begin{itemize}
% 	 	\item 熟练掌握Real-time PCR,Western Blot,细胞培养,质粒转染,流式细胞术,免疫组织(细胞)化学,免疫荧光等实验技术。
% 	 	\item 掌握激光共聚焦显微镜拍摄技术(Leica SP5),\textbf{信息检索}等。对\textbf{ImageJ}的使用熟练。熟练掌握使用\textbf{Graphpad}科研作图,Photoshop组图等。
% 	 \end{itemize} 	
% 	}
% \cventry{2013--2014}{参与人}{ADAMTS13与脑出血}{}{}{
% 	\begin{itemize}
% 		\item 掌握动物实验造模,给药,取材,行为学等动物实验技术,独立开展Real-time PCR实验。
% 	\end{itemize}
% 	}
% \cventry{2011--2013}{实习生}{武汉大学附属中南医院(三级甲等)}{}{}{
% 	\begin{itemize}
% 		\item 轮转普外科,血液科,心内科等\textbf{临床科室},了解临床科室运行流程。
% 		\item 轮转检验科各专业组,掌握基本的检验仪器的使用,熟悉检验科日常流程。
% 	\end{itemize}
% 	}
% \cventry{2010--2012}{参与人}{武汉大学大学生创新实验项目}{}{}{
% 	\begin{itemize}
% 		\item 培养科学素质,掌握细胞培养等实验技术。
% 	\end{itemize} 	
% 	}

 \section{专业技能}
 \cvdoubleitem{编程语言}{Python, R}{深度学习}{tensorflow}
 \cvdoubleitem{办公}{Office}{数据分析}{tableau}
 
% \section{所获证书}
% \cvdoubleitem{英语}{四级}{生物信息学导论与方法}{Coursera认证}{}{}
% \cvdoubleitem{优秀学生干部}{武汉大学}{硕士生学业奖学金,三等奖学金}{复旦大学}

%\cvlistdoubleitem{计算机科学}{篮球}
% \section{业余爱好}
% \cvitem{计算机科学}{初级Python使用;基于WordPress利用虚拟主机搭建个人博客;$LeTeX$排版}
% \cvitem{篮球}{本科班级篮球队主力队员}
 
 \end{document}